\documentclass[a4paper, 11pt, oneside]{article}

\usepackage[utf8]{inputenc}
\usepackage[T1]{fontenc}
\usepackage[french]{babel}
\usepackage{array}
\usepackage{shortvrb}
\usepackage{listings}
\usepackage[fleqn]{amsmath}
\usepackage{amsfonts}
\usepackage{fullpage}
\usepackage{enumerate}
\usepackage{graphicx}             % import, scale, and rotate graphics
\usepackage{subfigure}            % group figures
\usepackage{alltt}
\usepackage{url}
\usepackage{indentfirst}
\usepackage{eurosym}
\usepackage{listings}
\usepackage{color}
\usepackage[table,xcdraw,dvipsnames]{xcolor}

% Change le nom par défaut des listing
\renewcommand{\lstlistingname}{Extrait de Code}

% Change la police des titres pour convenir à votre seul lecteur
\usepackage{sectsty}
\allsectionsfont{\sffamily\mdseries\upshape} 
% Idem pour la table des matière.
\usepackage[nottoc,notlof,notlot]{tocbibind} 
\usepackage[titles,subfigure]{tocloft} 
\renewcommand{\cftsecfont}{\rmfamily\mdseries\upshape}
\renewcommand{\cftsecpagefont}{\rmfamily\mdseries\upshape} 

\definecolor{mygray}{rgb}{0.5,0.5,0.5}
\newcommand{\coms}[1]{\textcolor{MidnightBlue}{#1}}

\lstset{
    language=C, % Utilisation du langage C
    commentstyle={\color{MidnightBlue}}, % Couleur des commentaires
    frame=single, % Entoure le code d'un joli cadre
    rulecolor=\color{black}, % Couleur de la ligne qui forme le cadre
    stringstyle=\color{RawSienna}, % Couleur des chaines de caractères
    numbers=left, % Ajoute une numérotation des lignes à gauche
    numbersep=5pt, % Distance entre les numérots de lignes et le code
    numberstyle=\tiny\color{mygray}, % Couleur des numéros de lignes
    basicstyle=\tt\footnotesize, 
    tabsize=3, % Largeur des tabulations par défaut
    keywordstyle=\tt\bf\footnotesize\color{Sepia}, % Style des mots-clés
    extendedchars=true, 
    captionpos=b, % sets the caption-position to bottom
    texcl=true, % Commentaires sur une ligne interprétés en Latex
    showstringspaces=false, % Ne montre pas les espace dans les chaines de caractères
    escapeinside={(>}{<)}, % Permet de mettre du latex entre des <( et )>.
    inputencoding=utf8,
    literate=
  {á}{{\'a}}1 {é}{{\'e}}1 {í}{{\'i}}1 {ó}{{\'o}}1 {ú}{{\'u}}1
  {Á}{{\'A}}1 {É}{{\'E}}1 {Í}{{\'I}}1 {Ó}{{\'O}}1 {Ú}{{\'U}}1
  {à}{{\`a}}1 {è}{{\`e}}1 {ì}{{\`i}}1 {ò}{{\`o}}1 {ù}{{\`u}}1
  {À}{{\`A}}1 {È}{{\`E}}1 {Ì}{{\`I}}1 {Ò}{{\`O}}1 {Ù}{{\`U}}1
  {ä}{{\"a}}1 {ë}{{\"e}}1 {ï}{{\"i}}1 {ö}{{\"o}}1 {ü}{{\"u}}1
  {Ä}{{\"A}}1 {Ë}{{\"E}}1 {Ï}{{\"I}}1 {Ö}{{\"O}}1 {Ü}{{\"U}}1
  {â}{{\^a}}1 {ê}{{\^e}}1 {î}{{\^i}}1 {ô}{{\^o}}1 {û}{{\^u}}1
  {Â}{{\^A}}1 {Ê}{{\^E}}1 {Î}{{\^I}}1 {Ô}{{\^O}}1 {Û}{{\^U}}1
  {œ}{{\oe}}1 {Œ}{{\OE}}1 {æ}{{\ae}}1 {Æ}{{\AE}}1 {ß}{{\ss}}1
  {ű}{{\H{u}}}1 {Ű}{{\H{U}}}1 {ő}{{\H{o}}}1 {Ő}{{\H{O}}}1
  {ç}{{\c c}}1 {Ç}{{\c C}}1 {ø}{{\o}}1 {å}{{\r a}}1 {Å}{{\r A}}1
  {€}{{\euro}}1 {£}{{\pounds}}1 {«}{{\guillemotleft}}1
  {»}{{\guillemotright}}1 {ñ}{{\~n}}1 {Ñ}{{\~N}}1 {¿}{{?`}}1
}
\newcommand{\tablemat}{~}

%%%%%%%%%%%%%%%%% TITRE %%%%%%%%%%%%%%%%
\newcommand{\intitule}{Projet 2 : Dictionnaires}
\newcommand{\GrNbr}{s180498-s170220}
\newcommand{\PrenomUN}{Martin}
\newcommand{\NomUN}{RANDAXHE}
\newcommand{\PrenomDEUX}{Cyril}
\newcommand{\NomDEUX}{RUSSE}
\renewcommand{\tablemat}{\tableofcontents}

\title{\textbf{INFO0250 - Programmation avancée} \\\intitule}
\author{Groupe \GrNbr : \PrenomUN~\textsc{\NomUN}, \PrenomDEUX~\textsc{\NomDEUX}}
\date{}
\begin{document}
\maketitle
\newpage
\tablemat
\newpage

%%%%%%%%%%%%%%%%%%%%%%%%%%%%%%%%%%%%%%%%%%%%%%%%
\begin{document}
\section{\textbf{Ensemble}}
\subsection{\textbf{Implémentation}}
(1.1.a)
\subsubsection{Arbre Binaire}
Pour les arbres binaires, nous avons choisi d'implémenter un AVL. Nous allons passer 
en revue certaines fonctions afin d'expliquer les choix que nous avons faits. \\

Tout d'abords, au niveau des structures, elles se comptent au nombre de 2.

La structure "Set" principale étant définie dans le .h. Celle-ci contient la racine 
de l'arbre et la taille. La taille n'est pas très utile mais étant donné qu'il 
nous a été demandé d'implémenter la fonction permettant de savoir le nombre 
d'élément dans l'abre, dès lors, nous l'incrémentons de 1 à chaque insertion.

Nous avons créé une seconde structure propre à cette implémentation : "noeud", 
qui correspond, comme son nom l'indique, à un noeud de l'arbre. Celle-ci contient 
des pointeurs vers ses fils et son parent(pour la racine il sera nul) et une variable 
permettant de savoir la taille max de ses sous-arbres, cette variable sera d'une 
importance cruciale lors du rééquilibrage. Elle contient évidemment la chaine de 
caractères étant l'élément inséré.

Pour la libération de mémoire, nous sommes tout simplement parti sur une fonction 
récursive que nous pourrions comparer à une fonction permettant de savoir la 
hauteur de l'abre comme nous avions vu dans le cours.\\

Pour ce qui est de l'insertion, nous cherchons l'emplacement où ajouter la nouvelle 
valeur en partant de la racine. Si celle-ci est nulle, nous le rajoutons à la racine, 
sinon nous l'ajoutons en tant que fils droit ou gauche du noeud parent que nous 
venons de trouver. Après cette insertion nous faisons appel aux fonctions 
d'équilibrage de l'arbre afin de rechercher un éventuel déséquilibre et de le 
corriger. Nous veillons évidemment aussi à mettre à jour la racine du set en cas 
de modification de celle-ci après avoir fait appel à l'équilibrage.\\

Nous en venons donc à la partie la plus importante de cet algorithme qui est 
l'équilibrage de l'arbre. Pour nous facilitez la tâche, nous avons, comme annoncé 
précédemment, pour chaque noeud une variable pour la taille max de ses sous-arbres. 
A chaque insertion, nous utilisons une fonction qui met à jour ces valeurs. Pour ce 
faire, nous faisons une remontée récursive depuis le nouveau noeud jusqu'à la racine 
et nous mettons à jour la variable de chaque noeud en comparons les valeurs de 
hauteurs de chaque fils, la nouvelle valeur de hauteur est dorénavant la hauteur du 
plus grand sous-arbre + 1.\\ 

Pour ce qui est de l'équilibrage en lui-même, à l'aide de ces hauteurs de sous-arbres 
nous faisons, comme pour l'ajustement des hauteurs, une remontée récursive comparant les 
hauteurs de sous arbres et dans le cas où leur différence est supérieur à 1 en valeur 
absolue nous définissons de quel type de désiquilibre nous avons affaire au vue de cette valeur 
et nous vérifions si nous sommes dans le cas d'un désiquilibre intérieur en vérifiant quels 
sous-arbre est le plus grand du côté du désiquilibre. Ensuite, nous appliquons des rotations 
adéquates à la situation. Les rotations s'occupent évidemment de mettre à jour les fils/parents 
de tous les noeuds nécessitant des changements.

\subsubsection{Table de Hash}

Comme pour notre arbre binaire, nous avons 2 structures, l'une étant l'implémentation 
de "Set" défini dans le fichier h, comprenant un tableau de liste chainée, la 
taille du tableau et le nombre d'éléments dans la table de hash. L'autre structure "Liste" 
est l'élément qui correspond aux cellules que nous allons utiliser pour faire nos 
listes chainées. Celle-ci contient uniquement la chaine de caractères étant 
l'élément inséré et un pointeur vers la cellule suivante de la liste chainée, étant 
nul si nous sommes au dernier élément de la liste.\\

Pour ce qui est de la création et de la destruction du set, il n'y a pas grand chose 
de particulier. Nous avons juste choisi, comme valeur faible pour la taille initiale 
de notre tableau, 64. \\

Notre choix de table de hash est basé sur une implémentation en adressage fermé. Ce qui 
veut dire que nous gérons les superpositions par le biais de listes chainées. Nous sommes 
parti du principe de vouloir promouvoir l'efficacité à la mémoire donc pour notre stratégie de 
hachage, nous procédons à un réhachage dans le cas où le nombre d'élément est égal à 
la taille du tableau ce qui correspondrait dans le meilleur des cas, à ce stade là,
à un tableau de n cases contenant chacune une liste de une cellule. Si nous dépassons cette limite, 
nous doublons la taille du tableau et nous replacons tous les éléments déjà insérés à leur 
place dans le nouveau tableau. 
Nous en venons donc à notre fonction de hachage. Celle-ci 
correspond à la formule suivante :
$$\sum_{i=0}^{n}(elem[i]*727^i)\%m$$ avec 
\begin{itemize}
  \item elem, la chaine caractère
  \item n, le nombre de caractère dans elem
  \item m, la taille du tableau de liste du set
\end{itemize}
D'où le fait que nous devons repasser tous les éléments déjà 
insérés dans la fonction de hachage avec la nouvelle taille de 
tableau, afin de revoir sa postion dans celui-ci.\\

Enfin, notre fonction "contains", permet de vérifier l'existence d'un élément 
en utilisant, l'utilité principale d'une table de hachage, l'accès direct. Nous passons donc 
l'élément dans notre fonction de hachage et vérifions les éléments(s'il y en a) à la position 
du tableau que nous a renvoyé notre fonction de hachage.

\subsection{\textbf{Fonction d'Intersection}}
(1.1.b)\\

\subsection{\textbf{ Complexité "SetIntersection"}}
(1.1.c && 1.1.d)
Etant donné les choix d'implémentation que nous avons fait et que 
nous détaillerons dans la seconde section nous n'utilisons pas la 
fonction "SetIntersection". En effet, au vue de notre choix nous n'avons pas besoin 
de créer 2 set ce qui est un gain et qui explique que nous ne produisons 
pas de fonction propre à chaque implémentation et qui ne nous permet pas 
d'utiliser ces fonctions étant donné que leur définition nous est donné 
avec 2 set en arguments. Nous utiliserons plutot la fonction GetIntersection 
directement dans StringArray.c pour établir les intersections.

\section{\textbf{Intersection de deux fichiers}}
\subsection{\textbf{Complexité minimale}}
(2.1)\\
La complexité minimale consisterait à trié un des tableaux donc $\Theta(n*log(n))$ et puis 
faire une recherche dichotomique pour chaque élément de l'autre tableau, également $\Theta(n*log(n))$. 
Donc au global, $\Theta(n*log(n))$ également.
\subsection{\textbf{Analyse d'implémentations}}
(2.2.a)\\
Première approche :
\begin{itemize}
  \item Arbre Binaire 
  Cela revient au même si ce n'est que si l'on prend en compte le temps pour créer 
  les arbres, cette approche à l'avantage de n'en créer qu'un.
  \item Table de Hash
  Cette approche a aussi l'avantage de ne créer qu'une seule table et la complexité 
  restera quadratique en pire cas.
\end{itemize}
(2.2.b)\\
Deuxième approche :
\begin{itemize}
  \item Arbre Binaire
  Avoir 2 arbres binaires ne nous apporterait pas grand chose. Tel est la technique que nous 
  avons utilisé, nous parcourons le premier arbre et cherchons dans le deuxième. Nous 
  obtenons, comme prouvé dans la précédente section, une complexité $\Theta(n*log(n))$.
  Un avantage aurait pu être de faire une implémentation se rapprochant du mergesort 
  comme expliqué au point 2.3 mais nous avons remarqué que pour parcourir l'abre de 
  manière croissante, nous devrions passé par une fonction permettant d'avoir le successeur 
  du noeud actuel et ces montées et descente dans l'abre revienne presqu'au même que les 
  descente que l'on fait lors de notre recherche par la fonction "contains".
  \item Table de Hash
  Pour la table de Hash, même chose, nous parcourerons l'entierté d'une d'entre elles 
  et le pire cas consiste en tous les éléments de la seconde dans une seule liste 
  qui implique une complexité quadratique.
\end{itemize}
\subsection{Cas moyen}
(2.2.c)\\
Les conclusions que nous avons faite pour le pire cas restent inchangées, et nos 
complexités correspondent aux complexités en cas moyen exprimée dans nos analyses 
au point 1.1.d.
\subsection{Pertinence d'approche}
(2.2.d)\\
De par les points précédents, il parait plus judicieux de ne créer qu'un seul ensemble. 
Car le fait d'en créer 2 n'apporte pas grand chose et implique de prendre le temps de 
créer 2 ensembles qui n'est pas gratuit en mémoire ni en temps. Pour ce qui est du 
type d'ensemble, l'arbre binaire est dans tous les cas plutot éfficaces avec une complexité 
$\Theta(n*log(n))$. Pour ce qui est de la table de Hash, en moyenne, c'est la plus 
efficace avec une complexité linéaire, mais elle a en pire cas, bien que peu probable, 
une complexité $\Theta(n^2)$.
\subsection{\textbf{Complexité avec 2 ensembles triés}}
(2.3)\\
Nous pourrions procéder d'une manière similaire au mergesort, en effet nous pourrions 
parcourir les 2 tableaux parallèlement et si l'on trouve un élément similaire on l'ajoute à notre 
tableau intersection. En terme de complexité, dans le meilleure cas nous ne parcourerions que la plus 
petit des 2 tableaux donc $N_A \Theta(n)$. En pire cas, ça serait le même cas qu'un merge de A et B, 
donc $N_A + N_B$ ce qui revient encore une fois à du $\Theta(n)$.





\end{document}